%%%%%%%%%%%%%%%%%%%%%%%%%%%%%%%%%%%%%%%%%
% Wenneker Article
% LaTeX Template
% Version 2.0 (28/2/17)
%
% This template was downloaded from:
% http://www.LaTeXTemplates.com
%
% Authors:
% Vel (vel@LaTeXTemplates.com)
% Frits Wenneker
%
% License:
% CC BY-NC-SA 3.0 (http://creativecommons.org/licenses/by-nc-sa/3.0/)
%
%%%%%%%%%%%%%%%%%%%%%%%%%%%%%%%%%%%%%%%%%

%----------------------------------------------------------------------------------------
%	PACKAGES AND OTHER DOCUMENT CONFIGURATIONS
%----------------------------------------------------------------------------------------

\documentclass[10pt, a4paper, twocolumn]{article} % 10pt font size (11 and 12 also possible), A4 paper (letterpaper for US letter) and two column layout (remove for one column)

\input{structure.tex} % Specifies the document structure and loads requires packages

%----------------------------------------------------------------------------------------
%	ARTICLE INFORMATION
%----------------------------------------------------------------------------------------

\title{Introduction to Nim} % The article title

\author{
	\authorstyle{Boitumelo Phetla} % Authors
	\newline\newline % Space before institutions
	\textsuperscript{1}\institution{General-purpose Programming Language}\\ % Institution 1
}

% Example of a one line author/institution relationship
%\author{\newauthor{John Marston} \newinstitution{Universidad Nacional Autónoma de México, Mexico City, Mexico}}

\date{\today} % Add a date here if you would like one to appear underneath the title block, use \today for the current date, leave empty for no date

%----------------------------------------------------------------------------------------

\begin{document}

\maketitle % Print the title

\thispagestyle{firstpage} % Apply the page style for the first page (no headers and footers)

%----------------------------------------------------------------------------------------
%	ABSTRACT
%----------------------------------------------------------------------------------------

\lettrineabstract{Nim is unique. It is multi-paradigm, general purpose programming language with syntax like Python. However, the lanaguage or the design of this programming language does not emphasize on Object Oriented Programming style/concept. The language follows its own programming styling, it is imperative that its syntax styling is kept as guided by Nim's reference manual. Nim focuses mainly on effiency, expressiveness, and elegance. Nim is much like any other programming language with features such as concurrency, parallelism, user-defined types, the standard library, and more. Nim also has Nim's specific features such as asynchronous input/output, metaprogramming, and the foreign function interface.}

%----------------------------------------------------------------------------------------
%	ARTICLE CONTENTS
%----------------------------------------------------------------------------------------

\section{What is Nim?}

\begin{figure}[hbt!]
	\includegraphics[width=0.25\linewidth]{Nim.png} % Figure image
	%\caption{Nim} % Figure caption
	\label{nim} % Label for referencing with \ref{bear}
\end{figure}

Impatience is my vice, I want to know I have the right tools for the job before I commit. So what does this mean?
\\
\\
\textbf{\href{https://exercism.io/tracks/nim/installation}{How to install Nim}}
\\
\begin{lstlisting}
$bash: brew install nim

==> Pouring nim-0.19.4.mojave.bottle.tar.gz
/usr/local/Cellar/nim/0.19.4: 411 files, 12.8MB
\end{lstlisting}
\newpage
\textbf{Simple Script}
\\
\\
Create file hello\_world.nim
\begin{lstlisting}
#simple comment line
echo "Hello world!"
\end{lstlisting}
\\
\\
\textbf{Compile Script}
\\
\\
Compile hello\_world.nim
\begin{lstlisting}
$bash: nim compile --run hello_world.nim
Hint: used config file '/usr/local/Cellar/nim/0.19.4/nim/config/nim.cfg' [Conf]
Hint: system [Processing]
Hint: hello_world [Processing]
CC: hello_world
CC: stdlib_system
Hint:  [Link]
Hint: operation successful (12382 lines compiled; 0.660 sec total; 16.383MiB peakmem; Debug Build) [SuccessX]
Hint: /Volumes/pulse/100/Nim/ch01/code/hello_world  [Exec]
Hello world!
\end{lstlisting}
\\
\textbf{Terminal output results}
\\
At this stage I try to understand what the terminal is showing and if I cannot make sense of it, I at least try to find out if my output results is as expected.
\begin{lstlisting}
Hint: used config file '/usr/local/Cellar/nim/0.19.4/nim/config/nim.cfg' [Conf]
		<---- :accessing this directory shows
			  |------ nim.cfg
			  |------ nimdoc.cfg
			  |------ nimdoc.tex.cfg
Hint: system [Processing]
Hint: hello_world [Processing]
CC: hello_world 	<--- looks like Nim is using clang
CC: stdlib_system	<--- looks like Nim is using clang
Hint:  [Link]
Hint: operation successful (12382 lines compiled; 0.660 sec total; 16.383MiB peakmem; Debug Build) [SuccessX] <--- processed Nim packages
Hint: /Volumes/pulse/100/Nim/ch01/code/hello_world  [Exec] <--- script location
Hello world! <--- Expected output
\end{lstlisting}

\textbf{What is in the file/script}

\begin{lstlisting}
In the code repo Nim has created some sort of an object file hello_world (machine code) that Nim (JVM if it is running on it) uses to compile and read the script. Languages such as Java, C, C++ uses such approach (compiled languages). Python does not do such (interpreted language).

$bash: tree -L 1
		hello_world
		hello_world.nim

0 directories, 2 files
\end{lstlisting}

%------------------------------------------------

\subsection{Beginning to learn Nim}

Nim is still a relatively new programming language (first scribbed books ---- Nim in Action, Dominik Picheta).

What should you know going in:
\begin{itemize}
	\item The language is not fully complete
	\item Nim is a general-purpose programming language
	\item effiency, expressiveness and elegance are Nim's standardised priority markers and they rank according to the order mentioned
	\item Nim shares many of Python's characteristics
	\item Nim is a compiled language (translated to C first)
	\item Nim is well suited for systems programming (hardware, OSs, IoT, etc)
	\item Nim is one of the few languages that uses its own language to interpret itself
	\item type system, execution model
	\item Applications that perform I/O operations, such as reading files or sending data over a network, are also well supported by Nim.
	\item Web applications (web frameworks like Jester)
	\item Nim can compile JavaScript
	\item Things that you might be familiar with are covered in Nim (procedures, methods, iterators, generics and templates)
	\item \href{https://github.com/nim-lang/Nim#contributing}{Nim's documentation}
\end{itemize}

\newpage
\subsection{Why should you use Nim}
\begin{description}
	\item[Python, C, C++, ObjC, JavaScript] $\longrightarrow$ Nim's design follows that of Python with the capability to perform foreign interfacing (C, C++, ObjectiveC)
	\item[Nim projected started in 2005] $\longrightarrow$  The language is 14 years old, so you will be some sort of an early adopter
	\item[Garbage collector] $\longrightarrow$  Garbage collection + manual memory management
	\item[Game developers] $\longrightarrow$  because of a garbage collector that can be turned on and off this is a useful application for GameDevs
	\item[Scientific computing] $\longrightarrow$  Data Scientists
	\item[Scripting] $\longrightarrow$  Clue codes
	\item[Operating Systems] $\longrightarrow$  Supports Windows, Linux, Unix
	\item[effiency] $\longrightarrow$  Nim focues on compile-time mechanisms (runtime becomes effient)
	\item[Package manager] $\longrightarrow$   Nimble
	\item[Environments to use Nim] $\longrightarrow$  Web applications, Kernel
	\item[Andreas Rumpf] $\longrightarrow$  Andreas Rumpf is the designer of Nimrod programming language, which he develops in his spare time. He is a software engineer working at a top secret company and constantly attempts to create his own start-up which he will allow himself to program in Nimrod full-time. He has programmed in various programming languages over the years (including quite obscure ones) without being satisfied with any of them. Andreas Rumpf holds a degree in Computer Science which he obtained from the University of Kaiserslautern.
\end{description}

\href{https://github.com/nim-lang/Nim#contributing}{The compiler, standard library, and related tools are all open source and written in Nim. The project is available on GitHub, and everyone is encouraged to contribute. Contributing to Nim is a good way to learn how it works and to help with its development.}
%------------------------------------------------

\subsection{Core features of Nim}

\begin{enumerate}
	\item Metaprogramming $\longrightarrow$ customizing Nim
	\item Style-insensitive $\longrightarrow$ camelCase or snake\_case
	\item Compilation to C $\longrightarrow$ enhances the language's performance
\end{enumerate}

Aliquam elementum nulla at arcu finibus aliquet. Praesent congue ultrices nisl pretium posuere. Nunc vel nulla hendrerit, ultrices justo ut, ultrices sapien. Duis ut arcu at nunc pellentesque consectetur. Vestibulum eget nisl porta, ultricies orci eget, efficitur tellus. Maecenas rhoncus purus vel mauris tincidunt, et euismod nibh viverra. Mauris ultrices tellus quis ante lobortis gravida. Duis vulputate viverra erat, eu sollicitudin dui. Proin a iaculis massa. Nam at turpis in sem malesuada rhoncus. Aenean tempor risus dui, et ultrices nulla rutrum ut. Nam commodo fermentum purus, eget mattis odio fringilla at. Etiam congue et ipsum sed feugiat. Morbi euismod ut purus et tempus. Etiam est ligula, aliquam eget porttitor ut, auctor in risus. Curabitur at urna id dui lobortis pellentesque.

\begin{table}
	\caption{Example table}
	\centering
	\begin{tabular}{llr}
		\toprule
		\multicolumn{2}{c}{Name} \\
		\cmidrule(r){1-2}
		First Name & Last Name & Grade \\
		\midrule
		John & Doe & $7.5$ \\
		Richard & Miles & $5$ \\
		\bottomrule
	\end{tabular}
\end{table}

%------------------------------------------------

\section{Section}

\begin{figure}
	\includegraphics[width=\linewidth]{bear.jpg} % Figure image
	\caption{A majestic grizzly bear} % Figure caption
	\label{bear} % Label for referencing with \ref{bear}
\end{figure}

In hac habitasse platea dictumst. Vivamus eu finibus leo. Donec malesuada dui non sagittis auctor. Aenean condimentum eros metus. Nunc tempus id velit ut tempus. Quisque fermentum, nisl sit amet consectetur ornare, nunc leo luctus leo, vitae mattis odio augue id libero. Mauris quis lectus at ante scelerisque sollicitudin in eu nisi. Nulla elit lacus, ultricies eu erat congue, venenatis semper turpis. Ut nec venenatis velit. Mauris lacinia diam diam, ac egestas neque sodales sed. Curabitur eu diam nulla. Duis nec turpis finibus, commodo diam sed, bibendum erat. Nunc in velit ullamcorper, posuere libero a, mollis mauris. Nulla vehicula quam id tortor ornare blandit. Aenean maximus tempor orci ultrices placerat. Aenean condimentum magna vulputate erat mattis feugiat.

Quisque lacinia, purus id mattis gravida, sem enim fringilla erat, non dapibus est tellus pellentesque velit. Vivamus pretium sem quis leo placerat, at dignissim ex iaculis. Donec neque tortor, pharetra quis vestibulum id, tempus scelerisque mi. Cras in mattis est. Integer nec lorem rutrum, semper ligula bibendum, iaculis neque. Sed in nunc placerat, viverra dui in, fringilla sem. Sed quis rutrum magna, vitae pellentesque eros.

Praesent maximus mauris vitae nisl pulvinar, at tristique tortor aliquam. Etiam sit amet nunc in nulla vulputate sollicitudin. Aliquam erat volutpat. Praesent pharetra gravida cursus. Quisque vulputate lacus nunc. Integer orci ex, porttitor quis sapien id, eleifend gravida mi. Etiam efficitur justo eget nulla congue mattis. Duis commodo vel arcu a pretium. Aenean eleifend viverra nisl, nec ornare lacus rutrum in.

Vivamus pulvinar ac eros eu pellentesque. Duis nibh felis, sagittis sed lacus at, sagittis mattis nisi. Fusce ante dui, tincidunt in scelerisque ut, sagittis at magna. Fusce tincidunt felis et odio tincidunt imperdiet. Cras ut facilisis nisl. Aliquam vitae consequat metus, eget gravida augue. In imperdiet justo quis nulla venenatis accumsan. Aliquam aliquet consectetur tortor, at sollicitudin sapien porta sed. Donec efficitur mauris id rhoncus volutpat. Vestibulum ante ipsum primis in faucibus orci luctus et ultrices posuere cubilia Curae; Sed bibendum purus dapibus tincidunt euismod. Nullam malesuada ultrices lacus, ut tincidunt dolor. Etiam imperdiet quam eget elit tincidunt scelerisque. Curabitur ut ullamcorper dui. Cras gravida porta leo, ut lobortis nisl venenatis pulvinar. Proin non semper nulla.

Praesent pretium nisl purus, id mollis nibh efficitur sed. Sed sit amet urna leo. Nulla sed imperdiet sem. Donec ut diam tristique, faucibus ligula vel, varius est. In ipsum ligula, elementum vitae velit ac, viverra tincidunt enim. Phasellus gravida diam id nisl interdum maximus. Ut semper, tortor vitae congue pharetra, justo odio commodo urna, vel tempus libero ex et risus. Vivamus commodo felis non venenatis rutrum. Sed pulvinar scelerisque augue in porta. Sed maximus libero nec tellus malesuada elementum. Proin non augue posuere, pellentesque felis viverra, varius urna. Lorem ipsum dolor sit amet, consectetur adipiscing elit. Donec dignissim urna eget diam dictum, eget facilisis libero pulvinar.

Aliquam ex tellus, hendrerit sed odio sit amet, facilisis elementum enim. Suspendisse potenti. Integer molestie ac augue sit amet fermentum. Vivamus ultrices ante nulla, vitae venenatis ipsum ullamcorper sed. Phasellus gravida felis sapien, ac porta purus pharetra quis. Sed eget augue tellus. Nam vitae hendrerit arcu, id iaculis ipsum. Pellentesque sed magna tortor.

In ac tempus diam. Sed nec lobortis massa, suscipit accumsan justo. Quisque porttitor, ligula a semper euismod, urna diam dictum sem, sed maximus risus purus sit amet felis. Fusce elementum maximus nisi a mattis. Nulla vitae elit erat. Integer sit amet commodo risus, eget elementum nulla. Donec ultricies erat sit amet sem commodo iaculis. Donec euismod volutpat lacus, ut tempor est lacinia a. Vivamus auctor condimentum tincidunt. Praesent sed finibus urna. Sed pellentesque blandit magna et rhoncus.

Integer vel turpis nec tellus sodales malesuada a vel odio. Fusce et lectus eu nibh rhoncus tempus vel nec elit. Suspendisse commodo orci velit, lacinia dictum odio accumsan et. Vivamus libero dui, elementum vel nibh non, fermentum venenatis risus. Aliquam sed sapien ac orci sodales tempus a eget dui. Morbi non dictum tortor, quis tincidunt nibh. Proin ut tincidunt odio.

Pellentesque ac nisi dolor. Pellentesque maximus est arcu, eu scelerisque est rutrum vitae. Mauris ullamcorper vulputate vehicula. Praesent fermentum leo ac velit accumsan consectetur. Aliquam eleifend ex eros, ut lacinia tellus volutpat non. Pellentesque sit amet cursus diam. Maecenas elementum mattis est, in tincidunt ex pretium ac. Integer ultrices nunc rutrum, pretium sapien vitae, lobortis velit.

%----------------------------------------------------------------------------------------
%	BIBLIOGRAPHY
%----------------------------------------------------------------------------------------

\printbibliography[title={Bibliography}] % Print the bibliography, section title in curly brackets

%----------------------------------------------------------------------------------------

\end{document}
